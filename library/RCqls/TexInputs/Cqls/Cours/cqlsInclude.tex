\usepackage{mathtools}
\usepackage{amssymb}
%\VV replaced by \Var since \VV already exists

\newcommand{\cqlsbm}[1]{\mbox{\boldmath$#1$}}

\newcommand{\cqlshat}[1]{\widehat{#1}} 
%but modified in mode website in utils.dyn

\newcommand{\dataEmp}[2][n]{
{\left({#2}\right)}_{#1}
}

\newcommand{\meanEmp}[2][n]{
\overline{\left({#2}\right)}_{#1}
}

\newcommand{\sdEmp}[2][n]{
\overleftrightarrow{\left({#2}\right)}_{#1}
}

\newcommand{\quantEmp}[3][n]{
q_{#3}\left( \left( #2\right)_{#1} \right)
}

\newcommand{\quant}[2]{
q_{#2}\left( {#1} \right)
}






\newcommand{\Var}{
        \mathbb{V}\mbox{ar}
}
%\CC replaced by \Cov
\newcommand{\Cov}{
        \mathbb{C}ov
}

\newcommand{\Mat}[1]{
  \underline{\cqlsbm{#1}}
}

\newcommand{\Head}[1]{\centerline{\fbox{#1}}}

\newcommand{\tr}[1]{
 {#1}^{\!  T}
}

%\Vec replaced by \Vect since \Vec already exists
\newcommand{\Vect}[1]{
  \cqlsbm{#1}
}

\newcommand{\Esp}{
  \cqlsbm{E}
}

\newcommand{\FBox}[1]{
  \[ \mbox{ \fbox{$#1$}} \]
}

\newcommand{\EqBox}[2]{
\centerline{\fbox{\begin{minipage}{#1}\vspace*{-.4cm}
#2
\end{minipage}}}
}

\newcommand{\Proj}{
  \underline{\cqlsbm{\mathcal{P}}}
}

\newcommand{\EV}{
        \mathcal{L}
}


\newcommand{\Def}{
  \stackrel{\tiny \mbox{D{\'e}f.}}{=}
}

\newcommand{\Prob}[1]{
 \mathbb{P}\left( #1 \right)
}

\newcommand{\ProbH}[2]{
 \mathbb{P}_{#1} \left( #2 \right)
}

\newcommand{\NotR}{
\ensuremath{\stackrel{\texttt{R}}{=}}
}

\newcommand{\Not}{
  \stackrel{\tiny \mbox{Not.}}{=}
}

\newcommand{\From}[1]{
\! \left(\Vect #1\right)
}

\newcommand{\Dep}[2]{
\!  \left(\Vect #1 | \Mat #2\right)
}

\newcommand{\U}[2]{  % nouvelle notation 2003!
  u_{\scriptscriptstyle  #2}\!
}

\newcommand{\Del}[2]{  % nouvelle notation 2003-2004! et rechang� en 2006
  \cqlshat{\delta}_{\scriptscriptstyle #1,#2}\!
}

\newcommand{\DelLim}[2][\alpha]{  % nouvelle notation  2006
  \delta^{#2}_{lim,\scriptscriptstyle #1}\!
}

\newcommand{\Ex}[2]{
  \noindent \underline{\textbf{Exemple #1~:}}
   #2 
   \relax
}



\newcommand{\Com}[1]{
  \noindent \underline{Commentaires de l'{\'e}tudiant~:} \textit{#1} \\
}

\newcommand{\Rem}[1]{
  \noindent \underline{\textbf{Remarque~:}}#1 \\
}

\newcommand{\Cas}[1]{
  \noindent \underline{\textbf{Cas particulier d'un seul r{\'e}gresseur ($p=1$)~:}}#1 \\
}


\newcommand{\Est}[2]{
  \cqlshat{#1}\left(\cqlsbm{#2}\right)
}

\newcommand{\Int}[3]{
  \widetilde{{#1}}_{#2}\left(\cqlsbm{#3}\right)
}


\newcommand{\PPP}[1]{
 \mathbb{P}\left( #1 \right)
}


\newcommand{\EEE}[1]{
 \mathbb{E}\left( #1 \right) 
}

\newcommand{\VVV}[1]{
 \mathbb{V}ar\left( #1 \right) 
}

\newcommand{\Indic}{
        \large 1
}

\newcommand{\PR}[1]{
\textit{Produit #1}
}

\newcommand{\POP}[1][\bullet]{\ensuremath{\cqlsbm{\mathcal{Y}}^{#1}}}


\def\RR{\hbox{I\kern-.1em\hbox{R}}}
\def\NN{\hbox{I\kern-.1em\hbox{N}}}
\def\ZZ{\hbox{Z\kern-.3em\hbox{Z}}}
\def\PP{\hbox{P\kern-.8em\hbox{P}}}

\newtheorem{thm}{Th�or�me}
\newtheorem{prop}{Proposition}
\newtheorem{dfn}{D�finition}
\newtheorem{rmq}{Remarque}
\newtheorem{lm}{Lemme}
\newtheorem{exercice}{Exercice}
