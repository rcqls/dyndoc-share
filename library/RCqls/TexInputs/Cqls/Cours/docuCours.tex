%%%%%%%%%%%%%%%%%%%%%%%%%%%%%%%%%%%%%%%%%%%%%%%%%%%%%%%%%%%%%%%%%%%%%%%%%%%%%%%%%
%          same as inputCours but with normal page in pdf format                % 
%%%%%%%%%%%%%%%%%%%%%%%%%%%%%%%%%%%%%%%%%%%%%%%%%%%%%%%%%%%%%%%%%%%%%%%%%%%%%%%%%
% Documentation (sommaire) :
% Ce document compil� � partir de pdflatex g�n�re un document dynamique pdf au format transparent 
% Ce m�me document compil� avec latex g�n�re un document dvi en double colonne dans le but de l'imprimer.
% Ceci est rendu possible en suivant les instructions suivantes :
% - placer la commande  ``\input inputCours.tex'' en pr�ambule du document tex
% - juste apr�s (resp. avant) \begin{document} (resp. \end{document}) inclure 
% la commande ``\begin{Cours}'' (resp. ``\end{Cours}''.
%
% - Description des commandes suivantes :
% - \QueTransparent{} (resp. \QuePapier{}) : sorties exclusives transparent pdf (resp. document dvi)
% - \NewPage remplace le r�le de \newpage dans le but de ne g�n�rer le saut de page que dans le cas 
% de document transparent (pdf)
% - \Cible{nomCible}, \Lien{nomCible}{texteInteractif} sont utilis�s pour faire des hyperliens (nomCible et 
% texteInteractif) �tant bien entendu � renommer pour chaque hyperlien.
% - \Boite{\ldots} ,\BoiteCentre{\ldots} (i.e. \centerline{\Boite{\ldots}): permettent de mettre en valeur une ligne 
% ou une partie de ligne.
% - \BoiteGrosse{\ldots} : correspond � BoiteCentre{\minipage{10cm}{\ldots}\end{minipage})
% - RougeClair, VertClair, Vert, BleuClair, RougeFonce, VertFonce, BleuFonce sont des couleurs utilisables 
% pour la d�finition des commandes ci-dessous.
% - \BoiteCouleur{couleur}{\ldots}, \BoiteCentreCouleur{couleur}{\ldots}, \BoiteGrosseCouleur{couleur}{\ldots} : extensions 
% coloris�es de  \Boite{\ldots} ,\BoiteCentre{\ldots} et \BoiteGrosse{\ldots}
% - \BoiteCouleurTaille{couleur}{taille}{\ldots} : �quivalent de \BoiteGrosseCouleur{couleur}{\ldots} avec possibilit� 
% de changement de la taille 10cm par ``taille''.
%
% Autre remarque : les packages suivants sont appel�s par cet input :
% - ifthen, geometry,multicol, lscape, mathenv, color, graphicx et hyperref
% dans cet ordre. 
%%%%%%%%%%%%%%%%%%%%%%%%%%%%%%%%%%%%%%%%%%%%%%%%%%%%%%%%%%%%%%%%%%%%%%%%%%%%%%%%%
\usepackage{ifthen}
\ifthenelse{\isundefined{\pdfoutput}}{\def\papier{}}{}
\ifthenelse{\isundefined{\papier}}
{\usepackage[
pdftex=true,
%%paperheight=9cm,
%%paperwidth=12cm,
nohead,
nofoot,
left=2cm,
right=2cm
]{geometry}
}{
\usepackage[
a4paper,
%landscape,
nohead,
nofoot,
left=2cm,
right=2cm
]
{geometry}
\usepackage{multicol}
\setlength\columnseprule{.4pt}
\usepackage{lscape}

}


\usepackage{mathenv,amsmath,amssymb}

\ifthenelse{\isundefined{\papier}}
{\def\linkcolor{RougeFonce}}{\def\linkcolor{blue}}
%include pour cours20001
\usepackage{color} %pour d�finir les couleurs
\usepackage{graphicx}
\usepackage[colorlinks=true,linkcolor=\linkcolor]{hyperref} %� d�commenter pour pdf
%\input tcilatex.tex

%%%% Macros

\newcommand{\PR}[1]{\textit{Produit~{#1}}}

\def\RR{\hbox{I\kern-.1em\hbox{R}}}
\def\NN{\hbox{I\kern-.1em\hbox{N}}}
\def\ZZ{\hbox{Z\kern-.3em\hbox{Z}}}
\def\PP{\hbox{P\kern-.8em\hbox{P}}}

\newtheorem{thm}{Th�or�me}
\newtheorem{prop}{Proposition}
\newtheorem{dfn}{D�finition}
\newtheorem{rmq}{Remarque}
\newtheorem{lm}{Lemme}
\newtheorem{exercice}{Exercice}


%definition de couleur
%\definecolor{White}{rgb}{1,1,1}
\definecolor{RougeClair}{rgb}{1,.9,.9}
\definecolor{VertClair}{rgb}{.9,1,.9}
\definecolor{Vert}{rgb}{.4,.8,.4}
\definecolor{BleuClair}{rgb}{.9,.9,1}
\definecolor{Titrepale}{rgb}{.95,.95,.1}
\definecolor{RougeFonce}{rgb}{.9,0,0}
\definecolor{VertFonce}{rgb}{0,.4,.0}
\definecolor{BleuFonce}{rgb}{0,0,.5}

\newcommand{\heading}[1]{
  \begin{center}
    \large\bf
    \fcolorbox{black}{Titrepale} {#1}
  \end{center}
\vspace{1ex minus 1ex}}

\newcommand{\Couleur}[2]{{\color{#1} #2}}
\newcommand{\CouleurGras}[2]{\textbf{\color{#1} #2}}
\newcommand{\BoiteCentre}[1]
{
  \centerline{\fcolorbox{RougeFonce}{BleuClair}{#1}}
}

\newcommand{\Boite}[1]
{
  \fcolorbox{RougeFonce}{VertClair}{#1}
}

\newcommand{\BoiteCentreCouleur}[2]
{
  \centerline{\fcolorbox{RougeFonce}{#1}{#2}}
}

\newcommand{\BoiteCouleur}[2]
{
  \fcolorbox{RougeFonce}{#1}{#2}
}

\newcommand{\BoiteGrosseCouleur}[2]
{
\BoiteCentreCouleur{#1}{\begin{minipage}{10cm}{#2}\end{minipage}}
}

\newcommand{\BoiteGrosse}[1]
{
\BoiteCentre{\begin{minipage}{10cm}{#1}\end{minipage}}
}

\newcommand{\BoiteCouleurTaille}[3]
{
\BoiteCentreCouleur{#1}{\begin{minipage}{#2}{#3}\end{minipage}}
}

\newcommand{\Lien}[2]{\hyperlink{lien:#1}{#2}}

\newcommand{\Cible}[1]{\hypertarget{lien:#1}{}}

\ifthenelse{\isundefined{\papier}}{
\newcommand{\Retour}[1]{ \begin{flushright}\BoiteCouleur{RougeClair}{\hyperlink{lien:#1}{{\small {\color{black}Retour}}}}\end{flushright}}
}{
\newcommand{\Retour}[1]{}
}

\ifthenelse{\isundefined{\papier}}{
\newcommand{\NewPage}{\newpage}}
{\newcommand{\NewPage}{\bigskip \hrulefill}}

\newcommand{\QuePapier}[1]{
\ifthenelse{\isundefined{\papier}}{}{#1}}
\newcommand{\QueTransparent}[1]{
\ifthenelse{\isundefined{\papier}}{#1}{}}


\newcommand{\FigureTranspPapier}[3]{
\ifthenelse{\isundefined{\papier}}{\includegraphics[width=#1,height=#2]{#3.png}}{ \includegraphics[width=#1,height=#2]{#3.eps}}}




\def\DebutCours{\ifthenelse{\isundefined{\papier}}{}{\begin{landscape}\begin{multicols*}{2}  }}
\def\FinCours{\ifthenelse{\isundefined{\papier}}{}{\end{multicols*}\end{landscape}}}

\newenvironment{Cours}{\DebutCours}{\FinCours}