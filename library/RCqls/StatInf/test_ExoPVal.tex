
\documentclass[]{article}
%Packages
\usepackage{multirow}
\usepackage{graphicx}
\usepackage[utf8x]{inputenc}
\usepackage[T1]{fontenc}
\usepackage{aeguill}
\usepackage{amssymb}
\usepackage{xcolor}
\usepackage[a4paper,pdftex=true]{geometry}
\usepackage{fancyvrb}

%Preamble
\usepackage{tikz}


\definecolor{beaubleu}{rgb}{0.26,0.31,0.61}
\definecolor{beauvert}{rgb}{0.27,0.52,0.42}

%Styles

%Title

\begin{document}





Le professeur de statistique tape dans son logiciel préféré (à préciser) les quatre instructions \texttt{R} suivantes calculant des \textbf{p$-$valeurs} associées à quelques questions particuli{\`e}res qu'il se pose~:


\begin{Verbatim}[commandchars=\\\{\},fontfamily=courier,fontseries=b,fontsize=\small]
> \textcolor{beaubleu}{# test 1}
> \textcolor{beaubleu}{2*(1-pnorm((var(yExamC)-var(yExamD)-0)/seDVar(yExamC,yExamD)))}
\textit{\textcolor{beauvert}{Erreur dans evalq(\{ : objet 'yExamC' introuvable}}
> \textcolor{beaubleu}{# test 2}
> \textcolor{beaubleu}{1-pnorm((mean(yExamC)-mean(yExamD)-1)/seDMean(yExamC,yExamD))}
\textit{\textcolor{beauvert}{Erreur dans evalq(\{ : objet 'yExamC' introuvable}}
> \textcolor{beaubleu}{# test 3}
> \textcolor{beaubleu}{2*(1-pnorm((var(yExamC)-var(yExamD)-0)/seDVar(yExamC,yExamD)))}
\textit{\textcolor{beauvert}{Erreur dans evalq(\{ : objet 'yExamC' introuvable}}
> \textcolor{beaubleu}{# test 4}
> \textcolor{beaubleu}{1-pnorm((mean(yExamC)-mean(yExamD)-1)/seDMean(yExamC,yExamD))}
\textit{\textcolor{beauvert}{Erreur dans evalq(\{ : objet 'yExamC' introuvable}}
\end{Verbatim}



A la seule lecture de ces quatre instructions compl{\'e}tez le tableau suivant (\textbf{SANS} justification).\\
\hspace*{-1.3cm}\begin{tabular}{|c|c|c|c|}\hline
Test  & hypoth{\`e}se $H_1$ & Expression litt{\'e}rale de $H_1$ & {\small{Acceptation de $H_1$ }} \\
\hline \hline
<pre style='background-color: #EBECE4;'><code>Dyn Runtime Error
=> Leaving block depth 6: 
[:blck, :"=", [:named, "test", [:main, ":.test.3"]], :>, [:main, ""], [:for, [:args, [:main, "col in 0..3\n"]], :>, [:named, "result", [:main, ""], [:case, [:args, [:main, ":{col}\n"]], :when,  ...... [:args, [:main, ":{col}!=3"]], [:blck, :>, [:main, " & "]]], [:main, ""]], [:main, "\\\\\\hline\n"]]
</code></pre><pre style='background-color: #EBECE4;'><code>Dyn Runtime Error
=> Leaving block depth 6: 
[:blck, :"=", [:named, "test", [:main, ":.test.3"]], :>, [:main, ""], [:for, [:args, [:main, "col in 0..3\n"]], :>, [:named, "result", [:main, ""], [:case, [:args, [:main, ":{col}\n"]], :when,  ...... [:args, [:main, ":{col}!=3"]], [:blck, :>, [:main, " & "]]], [:main, ""]], [:main, "\\\\\\hline\n"]]
</code></pre><pre style='background-color: #EBECE4;'><code>Dyn Runtime Error
=> Leaving block depth 6: 
[:blck, :"=", [:named, "test", [:main, ":.test.3"]], :>, [:main, ""], [:for, [:args, [:main, "col in 0..3\n"]], :>, [:named, "result", [:main, ""], [:case, [:args, [:main, ":{col}\n"]], :when,  ...... [:args, [:main, ":{col}!=3"]], [:blck, :>, [:main, " & "]]], [:main, ""]], [:main, "\\\\\\hline\n"]]
</code></pre><pre style='background-color: #EBECE4;'><code>Dyn Runtime Error
=> Leaving block depth 6: 
[:blck, :"=", [:named, "test", [:main, ":.test.3"]], :>, [:main, ""], [:for, [:args, [:main, "col in 0..3\n"]], :>, [:named, "result", [:main, ""], [:case, [:args, [:main, ":{col}\n"]], :when,  ...... [:args, [:main, ":{col}!=3"]], [:blck, :>, [:main, " & "]]], [:main, ""]], [:main, "\\\\\\hline\n"]]
</code></pre>\end{tabular}


Associez {\`a} chacun des tests (1 {\`a} 4) pr{\'e}c{\'e}dents le graphique repr{\'e}sentant la r{\`e}gle de d{\'e}cision trac{\'e}e au seuil de 5\% bas{\'e}e sur $\Est{\delta_{\theta,\theta_0}}{y}$ not{\'e}e \texttt{deltaEst.H0} en \texttt{R} (\textbf{AVEC} justification). 


\pgfdeclareimage[width=7cm,height=7cm,interpolate=true]{img1}{/Users/rcqls/GitHub/dyndoc-share/library/RCqls/StatInf/rsrc/test_ExoPVal.tex/img/rfig-img1.png}
\pgfdeclareimage[width=7cm,height=7cm,interpolate=true]{img2}{/Users/rcqls/GitHub/dyndoc-share/library/RCqls/StatInf/rsrc/test_ExoPVal.tex/img/rfig-img2.png}
\pgfdeclareimage[width=7cm,height=7cm,interpolate=true]{img3}{/Users/rcqls/GitHub/dyndoc-share/library/RCqls/StatInf/rsrc/test_ExoPVal.tex/img/rfig-img3.png}
\pgfdeclareimage[width=7cm,height=7cm,interpolate=true]{img4}{/Users/rcqls/GitHub/dyndoc-share/library/RCqls/StatInf/rsrc/test_ExoPVal.tex/img/rfig-img4.png}












\begin{pgfpicture}{0cm}{0cm}{15cm}{15cm}
\pgfputat{\pgfxy(0,7.5)}{\pgfbox[left,bottom]{\pgfuseimage{img1}}}
\pgfputat{\pgfxy(7.5,7.5)}{\pgfbox[left,bottom]{\pgfuseimage{img3}}}
\pgfputat{\pgfxy(0,0)}{\pgfbox[left,bottom]{\pgfuseimage{img2}}}
\pgfputat{\pgfxy(7.5,0)}{\pgfbox[left,bottom]{\pgfuseimage{img4}}}
\pgfputat{\pgfxy(3.75,14)}{\pgfbox[center,center]{ Graphique 1}}
\pgfputat{\pgfxy(11.25,14)}{\pgfbox[center,center]{ Graphique 2}}
\pgfputat{\pgfxy(3.75,6.5)}{\pgfbox[center,center]{ Graphique 3}}
\pgfputat{\pgfxy(11.25,6.5)}{\pgfbox[center,center]{ Graphique 4}}

\end{pgfpicture}







\end{document}



